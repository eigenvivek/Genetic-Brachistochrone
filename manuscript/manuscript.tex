\documentclass{article}
\usepackage{amsmath}

\title{Optimizing the Brachistochrone using a Genetic Algorithm}
\author{Vivek Gopalakrishnan}
\date{July 11, 2018}

\begin{document}
\maketitle

\section{Defining the Cost Function}
Find the time required for a particle to travel a path $y=f(x)$ with endpoints $(x_0,f(x_0), (x_n, f(x_n))$. Assume a uniform gravitational field and that there are no nonconservative forces. Additionally, assume the particle traveling the path starts at rest.

% Total energy
Because we are assuming a uniform gravitational field, the total energy of the system as a function of x is
\begin{equation}
	E(x) = K + U_g = \frac{1}{2}mv(x)^2 + mgh(x) = \frac{1}{2}mv(x)^2 + mgf(x) \,.
\end{equation}
Additionally, because the system starts at rest, $K(x_0)=0 \,.$
\noindent
% Solve for v(x)
By combining these equations, we can derive the following:
\begin{equation}
	\begin{split}
		v(x)^2 &= 2g(f(x_0) - f(x)) \,, \\
		v(x) &= \sqrt{2g(f(x_0) - f(x))} \,.
	\end{split}
\end{equation}

\noindent
% Solve for ds 
To find the differential displacement along the curve ($ds$) for a differential change in the x direction ($dx$), we can use the arc length formula:
\begin{equation}
	ds = dx \sqrt{1 + f'(x)^2} \,.
\end{equation}

% Solve for T
The time required to travel a differential portion of $f(x)$ is given by $dt = \frac{ds}{v(x)}$. Thus, the total time $T$ required is as follows:
\begin{equation}
	\begin{split}
		T(f(x)) = \int{dt} = \int_{x_0}^{x_n} \frac{ds}{v(x)} &= \int_{x_0}^{x_n} \sqrt{\frac{1 + f'(x)^2}{2g(f(x_0) - f(x))}}dx \\
				        &\propto \int_{x_0}^{x_n} \sqrt{\frac{1 + f'(x)^2}{f(x_0) - f(x)}}dx \,.
	\end{split}
\end{equation}
\end{document}